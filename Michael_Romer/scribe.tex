%
% This is the LaTeX template file for lecture notes for EE 382C/EE 361C.
%
% To familiarize yourself with this template, the body contains
% some examples of its use.  Look them over.  Then you can
% run LaTeX on this file.  After you have LaTeXed this file then
% you can look over the result either by printing it out with
% dvips or using xdvi.
%
% This template is based on the template for Prof. Sinclair's CS 270.

\documentclass[twoside]{article}
\usepackage{graphics}
\setlength{\oddsidemargin}{0.25 in}
\setlength{\evensidemargin}{-0.25 in}
\setlength{\topmargin}{-0.6 in}
\setlength{\textwidth}{6.5 in}
\setlength{\textheight}{8.5 in}
\setlength{\headsep}{0.75 in}
\setlength{\parindent}{0 in}
\setlength{\parskip}{0.1 in}

%
% The following commands set up the lecnum (lecture number)
% counter and make various numbering schemes work relative
% to the lecture number.
%
\newcounter{lecnum}
\renewcommand{\thepage}{\thelecnum-\arabic{page}}
\renewcommand{\thesection}{\thelecnum.\arabic{section}}
\renewcommand{\theequation}{\thelecnum.\arabic{equation}}
\renewcommand{\thefigure}{\thelecnum.\arabic{figure}}
\renewcommand{\thetable}{\thelecnum.\arabic{table}}

%
% The following macro is used to generate the header.
%
\newcommand{\lecture}[4]{
   \pagestyle{myheadings}
   \thispagestyle{plain}
   \newpage
   \setcounter{lecnum}{#1}
   \setcounter{page}{1}
   \noindent
   \begin{center}
   \framebox{
      \vbox{\vspace{2mm}
    \hbox to 6.28in { {\bf EE 382V: Social Computing
                        \hfill Fall 2018} }
       \vspace{4mm}
       \hbox to 6.28in { {\Large \hfill Lecture #1: #2  \hfill} }
       \vspace{2mm}
       \hbox to 6.28in { {\it Lecturer: #3 \hfill Scribe: #4} }
      \vspace{2mm}}
   }
   \end{center}
   \markboth{Lecture #1: #2}{Lecture #1: #2}
   %{\bf Disclaimer}: {\it These notes have not been subjected to the
   %usual scrutiny reserved for formal publications.  They may be distributed
   %outside this class only with the permission of the Instructor.}
   \vspace*{4mm}
}

%
% Convention for citations is authors' initials followed by the year.
% For example, to cite a paper by Leighton and Maggs you would type
% \cite{LM89}, and to cite a paper by Strassen you would type \cite{S69}.
% (To avoid bibliography problems, for now we redefine the \cite command.)
% Also commands that create a suitable format for the reference list.
\renewcommand{\cite}[1]{[#1]}
\def\beginrefs{\begin{list}%
        {[\arabic{equation}]}{\usecounter{equation}
         \setlength{\leftmargin}{2.0truecm}\setlength{\labelsep}{0.4truecm}%
         \setlength{\labelwidth}{1.6truecm}}}
\def\endrefs{\end{list}}
\def\bibentry#1{\item[\hbox{[#1]}]}

%Use this command for a figure; it puts a figure in wherever you want it.
%usage: \fig{NUMBER}{SPACE-IN-INCHES}{CAPTION}
\newcommand{\fig}[3]{
			\vspace{#2}
			\begin{center}
			Figure \thelecnum.#1:~#3
			\end{center}
	}
% Use these for theorems, lemmas, proofs, etc.
\newtheorem{theorem}{Theorem}[lecnum]
\newtheorem{lemma}[theorem]{Lemma}
\newtheorem{proposition}[theorem]{Proposition}
\newtheorem{claim}[theorem]{Claim}
\newtheorem{corollary}[theorem]{Corollary}
\newtheorem{definition}[theorem]{Definition}
\newenvironment{proof}{{\bf Proof:}}{\hfill\rule{2mm}{2mm}}

% **** IF YOU WANT TO DEFINE ADDITIONAL MACROS FOR YOURSELF, PUT THEM HERE:
\usepackage{multirow}
\usepackage{qtree}
\usepackage{caption}

\begin{document}
%FILL IN THE RIGHT INFO.
%\lecture{**LECTURE-NUMBER**}{**DATE**}{**LECTURER**}{**SCRIBE**}
\lecture{16}{October 26}{Vijay Garg}{Michael Romer}
%\footnotetext{These notes are partially based on those of Nigel Mansell.}

% **** YOUR NOTES GO HERE:

% Some general latex examples and examples making use of the
% macros follow.  
%**** IN GENERAL, BE BRIEF. LONG SCRIBE NOTES, NO MATTER HOW WELL WRITTEN,
%**** ARE NEVER READ BY ANYBODY.
\section{Introduction} \label{intro}
For purposes of discussion, the following will be described from the perspective of two-player games. However, many of these concepts generalize to $n$-player games as well.

To put the following in the context of a concrete game, we first recap the two-player game where two students partnered together can each choose to either prepare for their joint presentation (strategy $P$) or prepare for their exam (strategy $E$). The payoffs of each student are given by the following bimatrix:

\begin{table}[h]
\centering
\renewcommand{\arraystretch}{1.5}
\begin{tabular}{cr|c|c|}
& \multicolumn{1}{c}{} & \multicolumn{2}{c}{Player 2} \\
& \multicolumn{1}{c}{} & \multicolumn{1}{c}{$P$} & \multicolumn{1}{c}{$E$} \\ \cline{3-4}
\multirow{2}*{Player 1}  & $P$ & $(98, 98)$ & $(94, 96)$ \\ \cline{3-4}
& $E$ & $(96, 94)$ & $(92, 92)$ \\ \cline{3-4}
\end{tabular}
\end{table}

\section{Best Response}
First we define the notion of a \textit{best response}, which is informally the best strategy a player can pick assuming they already know the strategy picked by the other player.

\begin{definition}[Best Response]
Let $\mu_i(s,t)$ be the payoff for player $i$ for a strategy profile $(s,t)$. A strategy $s$ for player~1 is a \textbf{best response~(BR)} to a strategy $t$ for player~2 if $\mu_1(s,t) \geq \mu_1(s',t)$ $\forall s' \neq s$. If $\mu_1(s,t) > \mu_1(s',t)$ $\forall s' \neq s$, then $s$ is a \textbf{strict best response} to t.
\end{definition}

In the game above, if either student knows their partner plans to prepare for the exam, then that student's best response would be to prepare for the presentation since their payoff would be 94 versus 92.

\section{Equilibrium}
Using the idea of best response, we are able to find certain equilibrium states within a given game. One common form of equilibrium named after the mathematician John Nash is the \textit{Nash equilibrium}.

\begin{definition}[Nash Equilibrium]
A strategy profile $(s,t)$ is a \textbf{Nash equilibrium} if strategy $s$ is a best response to strategy $t$ and if strategy $t$ is a best response to strategy $s$.
\end{definition}

In a Nash equilibrium, neither player has any motivation to deviate from their selected strategy. When extended to $n$-player games, a strategy profile $s=(s_1, s_2, \ldots, s_n)$ is a Nash equilibrium if every $s_i$ is a best response to $s_j$ for all $j \ne i$.

\subsection{Finding Nash Equilibria}
Consider the same game as described in section \ref{intro} except with the payoffs now given by the following bimatrix:

\begin{table}[h]
\centering
\renewcommand{\arraystretch}{1.5}
\begin{tabular}{cr|c|c|}
& \multicolumn{1}{c}{} & \multicolumn{2}{c}{Player 2} \\
& \multicolumn{1}{c}{} & \multicolumn{1}{c}{$P$} & \multicolumn{1}{c}{$E$} \\ \cline{3-4}
\multirow{2}*{Player 1}  & $P$ & $(90, 90)$ & $(86, 92)$ \\ \cline{3-4}
& $E$ & $(92, 86)$ & $(88, 88)$ \\ \cline{3-4}
\end{tabular}
\end{table}

To determine which of the strategy profiles is a Nash equilibrium, consider each profile and determine if any player can increase their payoff by changing only their own strategy.

\begin{itemize}
\item For strategy profile $(P, P)$, player 1 can increase their payoff from 90 to 92 by changing their strategy from $P$ to $E$. The same is similarly true for player 2, thus $(P, P)$ is \textit{not} a Nash equilibrium.

\item For strategy profile $(P, E)$, player 1 can increase their payoff from 86 to 88 by changing their strategy from $P$ to $E$. Similarly for strategy profile $(E, P)$, player 2 can increase their payoff from 86 to 88 by changing their strategy from $P$ to $E$. Thus, neither $(P, E)$ nor $(E, P)$ is a Nash equilibrium.

\item For strategy profile $(E, E)$, neither player 1 nor player 2 can increase their payoff by changing their strategy alone. Therefore, $(E, E)$ is the \textit{only} Nash equilibrium in this game.
\end{itemize}

\subsection{Trivial Game}
It is possible for all strategy profiles in a game to be Nash equilibria. One such example is the trivial game in which all payoffs are the same regardless of the action any player takes.

\begin{table}[h]
\centering
\renewcommand{\arraystretch}{1.5}
\begin{tabular}{cr|c|c|}
& \multicolumn{1}{c}{} & \multicolumn{2}{c}{Player 2} \\
& \multicolumn{1}{c}{} & \multicolumn{1}{c}{$A$} & \multicolumn{1}{c}{$B$} \\ \cline{3-4}
\multirow{2}*{Player 1}  & $A$ & $(1, 1)$ & $(1, 1)$ \\ \cline{3-4}
& $B$ & $(1, 1)$ & $(1, 1)$ \\ \cline{3-4}
\end{tabular}
\end{table}

\section{Mixed Strategies}
In the games considered up to this point, each player has selected as their strategy one action from a finite set of possible discrete actions. These strategies are known as \textit{pure strategies}. In some games, using a pure strategy is the worst possible strategy to use.

\subsection{Rock-Paper-Scissors}
Consider the game Rock-Paper-Scissors with the following payoff matrix:

\begin{table}[ht]
\centering
\renewcommand{\arraystretch}{1.5}
\begin{tabular}{cr|r|r|r|}
& \multicolumn{1}{c}{} & \multicolumn{3}{c}{Player 2} \\
& \multicolumn{1}{c}{} & \multicolumn{1}{c}{$R$} & \multicolumn{1}{c}{$P$} & \multicolumn{1}{c}{$S$} \\ \cline{3-5}
& $R$ & $(0, 0)$ & $(-1, 1)$ & $(1, -1)$ \\ \cline{3-5}
\multicolumn{1}{c}{Player 1} & $P$ & $(1, -1)$ & $(0, 0)$ & $(-1, 1)$ \\ \cline{3-5}
& $S$ & $(-1, 1)$ & $(1, -1)$ & $(0, 0)$ \\ \cline{3-5}
\end{tabular}
\end{table}

Suppose the game was setup such that, e.g., player 1 had to act first followed by player 2. If player 1 used a pure strategy, such as always picking rock, then player 2 would be able to win every time. In general, it is not possible for a player to play a \textit{deterministic pure strategy} such as this and \textit{not} have player order influence the game's outcome.

By considering only pure strategies, it is also possible to show that there exist no Nash equilibria in the Rock-Paper-Scissors game above. However, by allowing for randomized strategies where each player can choose an action probabilistically, it is possible for players to effectively act simultaneously. When players pick these so-called \textit{mixed strategies}, the game's outcome becomes independent of player order.

\begin{definition}[Mixed Strategy]
Let $S = \{s_1, s_2, \ldots, s_n\}$ be the finite set of pure strategies available to each player for a given game. A probability vector $\mathbf{s}=(p_{s_1}, p_{s_2}, \ldots, p_{s_n})$ is a \textbf{mixed strategy} for a player where $p_{s_i}$ is the probability of that player selecting the pure strategy $s_i$.
\end{definition}

Mixed strategies allow a player to pick their strategy at random rather than selecting a single pure strategy. In the game above, one possible mixed strategy is to select either rock, paper, or scissors with equal probability. The mixed strategy in this case corresponds with the probability vector $\left(\frac{1}{3},\frac{1}{3},\frac{1}{3}\right)$. The Nash equilibrium for the Rock-Paper-Scissors game is the strategy profile $\left(\left(\frac{1}{3},\frac{1}{3},\frac{1}{3}\right), \left(\frac{1}{3},\frac{1}{3},\frac{1}{3}\right)\right)$ where both players pick with equal likelihood either rock, paper, or scissors.

\subsection{Existence of Nash Equilibrium}
In his 1950 paper, John Nash showed that for every \textit{finite game}, that is any game with a finite number of players and finite number of strategies, at least one mixed strategy is guaranteed to exist that is also a Nash equilibrium. On the other hand, as was demonstrated in the Rock-Paper-Scissors game above, a Nash equilibrium is \textit{not} guaranteed to exist when only considering the set of possible pure strategies.

The following tree summarizes the two main classes of strategies discussed so far along with their respective results:

\Tree [.\textbf{Strategy} {\textbf{Pure} \\ \textit{Nash equilibrium may} \\ \textit{or may not exist.}} !\qsetw{3.5in} {\textbf{Mixed} \\ \textit{At least one Nash equilibrium guaranteed} \\ \textit{to exist for every finite game.}} ]

\subsection{Payoffs in Mixed Games}
To analyze any game, you need the bimatrix of payoffs. In a game with mixed strategies, the bimatrix lists the \textit{expected payoffs} rather than the absolute payoffs for each player. In this case, the mixed strategy profile corresponding to a Nash equilibrium will maximize each player's expected payoff given the strategies selected by all other players.

\section{A Tale of Two Firms}
In each of the following two-player games that follow, each player may only play a pure strategy. Mixed strategies are not allowed.

\subsection{Manufacturing Items}
Suppose two firms, Firm 1 and Firm 2, can employ only one of the following pure strategies:

\begin{itemize}
    \item Make low-priced items (strategy $L$)
    \item Make high-priced items (strategy $H$)
\end{itemize}

Additionally, we know the following information about the firm's target markets:

\begin{itemize}
    \item 60\% of the population will purchase exclusively low-priced items.
    \item If Firm 1 and Firm 2 compete with each other, then Firm 1 will win 80\% of the target market.
    \item If the two firms don't compete, they each get 100\% of their respective target markets.
\end{itemize}

This results in the following bimatrix of payoffs:

\begin{table}[h]
\centering
\renewcommand{\arraystretch}{1.5}
\begin{tabular}{cr|c|c|}
& \multicolumn{1}{c}{} & \multicolumn{2}{c}{Firm 2} \\
& \multicolumn{1}{c}{} & \multicolumn{1}{c}{$L$} & \multicolumn{1}{c}{$H$} \\ \cline{3-4}
\multirow{2}*{Firm 1}  & $L$ & $(.48, .12)$ & $(.60, .40)$ \\ \cline{3-4}
& $H$ & $(.40, .60)$ & $(.32, .08)$ \\ \cline{3-4}
\end{tabular}
\caption*{\textbf{NOTE:} This is an \textit{asymmetric} game.}
\end{table}

\subsubsection{Analysis}
Given the information above, the question is what will each firm do? When analyzing player strategies in any game, ask yourself the following questions:

\begin{itemize}
    \item Are there any strict dominant strategies?
    \item Are there any Nash equilibria?
\end{itemize}

If each player has a strict dominant strategy, then the optimal strategy profile that yields the greatest payoff for each player is easily determined. However, in this game only Firm~1 has a strict dominant strategy, which is to make low-priced items.

Even though Firm~2 has no strict dominant strategy of its own, it is able to see from the payoff matrix the strict dominant strategy that Firm~1 will follow. Given Firm~1's strategy will be to produce low-priced items, Firm 2's best response will be to produce high-priced items as this will yield a payoff of .40 as opposed to the .12 they would receive if they competed with Firm~1 and produced low-priced items.

Therefore, in this game, Firm~1 will produce low-priced items, and Firm~2 will produce high-priced items.

\subsection{Servicing Clients}
Now instead of manufacturing items, suppose Firm 1 and Firm 2 can each offer services to one of the three following clients: $A$, $B$, or $C$. Additionally, we know the following information:

\begin{itemize}
    \item If both firms pursue the same client, they split the business from that client evenly in half. The amount of business available from a given client depends on the size of that client.
    \item Firm 1 is too small to get business on its own from any client. If Firm~1 pursues a different client from Firm~2, then Firm~1's payoff will be zero.
    \begin{itemize}
        \item This means that in order to have a nonzero payoff, Firm 1 needs to coordinate with Firm 2.
    \end{itemize}
    \item Firm 2 can get the full business of either client B or client C on its own; however, client A is so big that if only one firm approaches it, then that firm's payoff will be zero.
\end{itemize}

The following bimatrix describes the above scenario along with giving the payoffs for each firm:

\begin{table}[h]
\centering
\renewcommand{\arraystretch}{1.5}
\begin{tabular}{cr|r|r|r|}
& \multicolumn{1}{c}{} & \multicolumn{3}{c}{Firm 2} \\
& \multicolumn{1}{c}{} & \multicolumn{1}{c}{$A$} & \multicolumn{1}{c}{$B$} & \multicolumn{1}{c}{$C$} \\ \cline{3-5}
& $A$ & $(4, 4)$ & $(0, 2)$ & $(0, 2)$ \\ \cline{3-5}
\multicolumn{1}{c}{Firm 1} & $B$ & $(0, 0)$ & $(1, 1)$ & $(0, 2)$ \\ \cline{3-5}
& $C$ & $(0, 0)$ & $(0, 2)$ & $(1, 1)$ \\ \cline{3-5}
\end{tabular}
\end{table}

\subsubsection{Analysis}
By looking at the bimatrix of payoffs, we can readily tell that no player in this game has a strict dominant strategy.

If no player has a strict dominant strategy, we then wish to identify any of the Nash equilibria in the game. In the absence of a strict dominant strategy, strategic players will pick strategies that form a Nash equilibrium because that would be the point at which none of the players have a motivation to deviate from their selected strategy.

In the game above, one can show that a Nash equilibrium exists at the strategy profile $(A, A)$. This also happens to be the game's \textit{social optimum}, which is the strategy profile that maximizes the \textit{social value}, i.e., the sum of the payoffs for that profile. For this particular game, the optimal social value is 8.

\end{document}





