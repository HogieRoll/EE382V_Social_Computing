%
% This is the LaTeX template file for lecture notes for EE 382C/EE 361C.
%
% To familiarize yourself with this template, the body contains
% some examples of its use.  Look them over.  Then you can
% run LaTeX on this file.  After you have LaTeXed this file then
% you can look over the result either by printing it out with
% dvips or using xdvi.
%
% This template is based on the template for Prof. Sinclair's CS 270.

\documentclass[twoside]{article}
\usepackage{graphics}
\setlength{\oddsidemargin}{0.25 in}
\setlength{\evensidemargin}{-0.25 in}
\setlength{\topmargin}{-0.6 in}
\setlength{\textwidth}{6.5 in}
\setlength{\textheight}{8.5 in}
\setlength{\headsep}{0.75 in}
\setlength{\parindent}{0 in}
\setlength{\parskip}{0.1 in}
\usepackage{empheq}
\usepackage{graphicx}
\usepackage{wrapfig}
\usepackage{amssymb}
\usepackage[utf8]{inputenc}
\usepackage[english]{babel}

%
% The following commands set up the lecnum (lecture number)
% counter and make various numbering schemes work relative
% to the lecture number.
%
\newcounter{lecnum}
\renewcommand{\thepage}{\thelecnum-\arabic{page}}
\renewcommand{\thesection}{\thelecnum.\arabic{section}}
\renewcommand{\theequation}{\thelecnum.\arabic{equation}}
\renewcommand{\thefigure}{\thelecnum.\arabic{figure}}
\renewcommand{\thetable}{\thelecnum.\arabic{table}}

%
% The following macro is used to generate the header.
%
\newcommand{\lecture}[4]{
   \pagestyle{myheadings}
   \thispagestyle{plain}
   \newpage
   \setcounter{lecnum}{#1}
   \setcounter{page}{1}
   \noindent
   \begin{center}
   \framebox{
      \vbox{\vspace{2mm}
    \hbox to 6.28in { {\bf EE 382V: Social Computing
                        \hfill Fall 2018} }
       \vspace{4mm}
       \hbox to 6.28in { {\Large \hfill Lecture #1: #2  \hfill} }
       \vspace{2mm}
       \hbox to 6.28in { {\it Lecturer: #3 \hfill Scribe: #4} }
      \vspace{2mm}}
   }
   \end{center}
   \markboth{Lecture #1: #2}{Lecture #1: #2}
   %{\bf Disclaimer}: {\it These notes have not been subjected to the
   %usual scrutiny reserved for formal publications.  They may be distributed
   %outside this class only with the permission of the Instructor.}
   \vspace*{4mm}
}

%
% Convention for citations is authors' initials followed by the year.
% For example, to cite a paper by Leighton and Maggs you would type
% \cite{LM89}, and to cite a paper by Strassen you would type \cite{S69}.
% (To avoid bibliography problems, for now we redefine the \cite command.)
% Also commands that create a suitable format for the reference list.
\def\beginrefs{\begin{list}%
        {[\arabic{equation}]}{\usecounter{equation}
         \setlength{\leftmargin}{2.0truecm}\setlength{\labelsep}{0.4truecm}%
         \setlength{\labelwidth}{1.6truecm}}}
\def\endrefs{\end{list}}
\def\bibentry#1{\item[\hbox{[#1]}]}

%Use this command for a figure; it puts a figure in wherever you want it.
%usage: \fig{NUMBER}{SPACE-IN-INCHES}{CAPTION}
\newcommand{\fig}[3]{
			\vspace{#2}
			\begin{center}
			Figure \thelecnum.#1:~#3
			\end{center}
	}
% Use these for theorems, lemmas, proofs, etc.
\newtheorem{theorem}{Theorem}[lecnum]
\newtheorem{lemma}[theorem]{Lemma}
\newtheorem{proposition}[theorem]{Proposition}
\newtheorem{claim}[theorem]{Claim}
\newtheorem{corollary}[theorem]{Corollary}
\newtheorem{definition}[theorem]{Definition}
\newenvironment{proof}{{\bf Proof:}}{\hfill\rule{2mm}{2mm}}

% **** IF YOU WANT TO DEFINE ADDITIONAL MACROS FOR YOURSELF, PUT THEM HERE:

\begin{document}
%FILL IN THE RIGHT INFO.
%\lecture{**LECTURE-NUMBER**}{**DATE**}{**LECTURER**}{**SCRIBE**}
\lecture{}{LP Formulation of the Stable Marriage Problem}{Vijay Garg}{Eric Addison}
%\footnotetext{These notes are partially based on those of Nigel Mansell.}

% **** YOUR NOTES GO HERE:

% Some general latex examples and examples making use of the
% macros follow.  
%**** IN GENERAL, BE BRIEF. LONG SCRIBE NOTES, NO MATTER HOW WELL WRITTEN,
%**** ARE NEVER READ BY ANYBODY.
\section{Introduction}
In this note, we will work out a linear programming (LP) formulation of the Stable Marriage problem. The problem will be briefly reviewed, followed by introduction of relevant notation. Characteristics of the problem will be translated into linear algebraic constraints, which will allow us to write down the LP statement of the problem, along with its dual (DLP). Finally, we will work through a small example. 

The formulation and notation in this note is based primarily on that found in \cite{DBLP:journals/mor/RothRV93}, however several other articles can easily be found that provide similar treatments. 

\section{Review of the Stable Marriage Problem}
Consider two sets of agents $M$ and $W$ (e.g. men and women), with $|M|=|W|$ for simplicity. Each agent provides a complete list of preferences of the members from the opposite set, which implicitly defines a ranking operator ($\geq_m$ for each man $m$ and $\geq_w$ for each woman $w$) such that $(M,\geq_w)$ and $(W,\geq_m)$ are totally ordered sets. We say that a woman $w$ prefers man $m$ to another man $m'$ if $m \geq_w m'$.

The goal of the stable marriage problem is to find a \textit{stable matching} between the two sets $M$ and $W$. Define a \textit{matching} as a mapping $\mu: (M \cup W) \rightarrow (M\cup W)$ where:
\begin{itemize}
    \item $\mu(w)=m \Leftrightarrow \mu(m)=w$, i.e. $m$ is matched to $w$
    \item if $\mu(w)\notin M$, then $\mu(w)=w$ and $w$ is not matched
    \item if $\mu(m)\notin W$, then $\mu(m)=m$ and $m$ is not matched 
\end{itemize}

A matching is \textit{stable} if $\nexists(m,w)$ such that $m >_w \mu(w)$ and $w >_m \mu(m)$ (where $>_m$ is the irreflexive version of $\geq_m)$. That is, there is no pair of matched man and matched woman who both prefer each other to their current match. Such a pair is know as a \textit{blocking pair}.

\section{LP Formulation}
We begin by defining the \textit{incidence vector} of a matching $\mu$ as a vector $x \in \lbrace 0, 1 \rbrace ^{|M|\times |W|}$, indexed such that $x_{m,w}=1 \Rightarrow \mu(m) = w$, else $x_{m,w}=0$. The following linear constraints are necessary and sufficient to classify an arbitrary $x$ as a matching:
\begin{align}
    \sum_{j\in W} x_{m,j} \leq 1 & \quad \forall m \in M \tag{A} \label{constraintA}\\
    \sum_{i\in M} x_{i,w} \leq 1 & \quad \forall w \in W \tag{B} \label{constraintB}\\
    x_{m,w} \geq 0 & \quad \forall (m,w) \in M \times W \tag{C} \label{constraintC}
\end{align}
Constraints (\ref{constraintA}) and (\ref{constraintB}) together define the requirement that a man (or woman) can be either unmatched or matched to a single other person: no double matchings. Constraint (\ref{constraintC}) is satisfied by the definition of $x$, but explicitly listing it will complete the linear program. 

If the nature of the incidence vector is unclear, imagine representing is as a matrix $X$, where each entry $X_{ij}$ is a boolean value that determines whether man $m_i$ is matched to woman $w_j$. Clearly to form a valid matching, we can have at most a single $1$ in each row and column, or else there would be a man or woman who is matched to more than one partner. A perfect matching is achieved when the matrix $X$ is a permutation matrix. 
\subsection{The Assignment Problem}

Before describing the linear program associated with the marriage problem (MP), we can first do so for the simpler assignment problem (AP), which contains only one sided preferences. In this problem, imagine that the set $M$ is a set of buyers, and $W$ is a set of sellers. Define a \textit{fractional matching} as a vector $x \in  \mathbb{R} ^{|M|\times |W|}$ that satisfies constraints (\ref{constraintA}-\ref{constraintC}). The linear program AP can then be written as 
\begin{align*}
\text{Maximize} \sum_{(i,j)\in M \times W} \alpha_{ij}x_{ij}\\
\text{subject to constraints \ref{constraintA}-\ref{constraintC}}
\end{align*}
where $\alpha_{ij}$ is the total value to both parties of a matching between $i$ and $j$. We know from linear optimization theory that the optimal solution(s) $x^*$ will lie at corners of the feasible region, which given these constraints will result in a solution in the form of the previously defined incidence vector, which takes values $x^* \in \lbrace 0, 1 \rbrace ^{|M|\times |W|}$. 

Unfortunately, this linear program is not in the standard form of \begin{align*}
    \text{Maximize } &\textbf{c}^{T}\textbf{x}\\
    \text{s.t. }&A\textbf{x} \leq \textbf{b}\\
    & x_i \geq 0
\end{align*}
from which we could easily write down the dual problem as:
\begin{align*}
    \text{Minimize } &\textbf{b}^{T}\textbf{y}\\
    \text{s.t. }&A^T\textbf{y} \geq \textbf{c}\\
    & y_i \geq 0
\end{align*}
Where we have switched from index notation to vector notation. How can we reform the constraints (\ref{constraintA}) and (\ref{constraintB}) to look like standard form? Start by referring to the incidence vector $\textbf{x}$ by a single index, i.e. $x_{i,j} \rightarrow x_k$, where $k = (i-1)N+j$. We can pick out the vector $\textbf{x}^{i}$, where $\textbf{x}^i_j = x_{i,j}$ for a particular man $m_i$ by multiplying $\textbf{x}$ by the following matrix:
\[
A = [\underbrace{\textbf{e}_1, \textbf{e}_1, ..., \textbf{e}_1}_N,\underbrace{\textbf{e}_2, \textbf{e}_2, ..., \textbf{e}_2}_N,......,\underbrace{\textbf{e}_N, \textbf{e}_N, ..., \textbf{e}_N}_N,]
\]

Where the basis vector $\textbf{e}_i \in \mathbb{R}^N$ has all zeros except for the $ith$ row, and $A \in \mathbb{R}^{N\times N^2}$. To be more explicit, the matrix $A$ looks like:
\[
A = \begin{bmatrix}
1 & 1 & ...& 1 & 0 & 0 & ... & 0 & 0 & ...\\
0 & 0 & ...& 0 & 1 & 1 & ... & 1 & 0 & ... \\
0 & 0 & ...& 0 & 0 & 0 & ... & 0 & 1 & ... \\
\vdots & \vdots & \vdots & \vdots & \vdots & \vdots & \vdots & \vdots & \vdots & \vdots
\end{bmatrix}
\]
where each block of 1's is $N$ elements long.

We can then rewrite constraint (\ref{constraintA}) as:
\[
A\textbf{x} \leq \textbf{1}_N
\]

where the vector of ones $\textbf{1}_N = [1, 1, ..., 1] \in \mathbb{R}^N$.

Similarly, we can rewrite constraint (\ref{constraintB}) using a matrix that picks out the woman vector $x_{i,w}$:
\[
B = [\textbf{e}_1, \textbf{e}_2, ..., \textbf{e}_N, \textbf{e}_1, \textbf{e}_2, ..., \textbf{e}_N, ...] = [\underbrace{I_N, I_N, ..., I_N}_N]
\]

Where $I_N$ is the $N\times N$ identity matrix. Now constraint (\ref{constraintB}) becomes:
\[
B\textbf{x} \leq \textbf{1}_N
\]

We can combine these into a single constraint by stacking the matrices:
\[
\begin{bmatrix} 
A \\
B
\end{bmatrix}\textbf{x} = D\textbf{x}\leq \textbf{1}_{2N} 
\]

Now we can write down the dual of the assignment problem as:
\begin{align*}
    \text{Minimize } &\textbf{1}_{2N}^{T}\textbf{y}\\
    \text{s.t. }&D^T\textbf{y} \geq \alpha\\
    & y_i \geq 0
\end{align*}

We can choose at this point to rewrite the vector $\textbf{y}$ as 
\[
\textbf{y} = \begin{bmatrix}
\textbf{u}\\
\textbf{v}
\end{bmatrix}
\]

which then changes the dual problem statement to
\begin{align*}
    \text{Minimize } &\textbf{1}_{N}^{T}\textbf{u} +\textbf{1}_{N}^{T}\textbf{v} \\
    \text{s.t. }&A^T\textbf{u} + B^T\textbf{v} \geq \alpha\\
    & u_i, v_j \geq 0
\end{align*}

or, moving back to index notation,
\begin{align*}
    \text{Minimize } &\sum_{i\in M}u_i + \sum_{j\in W}v_j\\
    \text{s.t. }&u_i + v_j \geq \alpha_{ij} \quad \forall (i,j) \in M \times W\\
    & u_i, v_j \geq 0
\end{align*}

As discussed previously in class, the dual variables $u_i$ and $v_j$ can be interpreted as node labels in a bipartite graph.

\subsection{The Marriage Problem}

The marriage problem, having two-sided preferences, requires an additional constraint to avoid blocking pairs:
\[
\sum_{j >_m w}x_{m,j} + \sum_{i >_w m}x_{i,w} + x_{m,w} \geq 1 \quad \forall (m,w) \in M \times W \tag{D} \label{constraintD}
\]
where the notation $j >_m w \equiv \lbrace j \in W : j >_m w \rbrace$ and $i >_w m \equiv \lbrace i \in M : i >_w m \rbrace$, for example, $j >_m w$ is the set of all women that are more preferable to $m$ than is $w$. Constraint (\ref{constraintD}) demands that at least one of three conditions are met:
\begin{itemize}
    \item $m$ is matched to a woman more desirable than $w$
    \item $w$ is matched to a man more desirable than $m$
    \item $m$ is matched to $w$
\end{itemize}

How does constraint (\ref{constraintD}) ensure there are no blocking pairs? Consider two matches $(m,w')$ and $(m',w)$, where $w >_m w'$ and $m >_w m'$, so that $(m,w)$ is a blocking pair. Then the pair $(m,w)$ violates constraint (\ref{constraintD}):
\[
\sum_{j >_m w}x_{m,j} + \sum_{i >_w m}x_{i,w} + x_{m,w} = 0+0+0=0
\]
because $m$ is not matched to a woman more desirable than $w$, $w$ is not matched to a man more desirable than $m$, and $m$ is not matched to $w$. Therefore, $\exists$ a blocking pair $(m,w) \Rightarrow$ constraint (\ref{constraintD}) is violated, so
\[
\text{Constraint (\ref{constraintD}) is satisfied } \Rightarrow \nexists \text{ any blocking pairs}.
\]

The linear program for the marriage problem (MP) can then be written as 
\begin{align*}
\text{Maximize} \sum_{(i,j)\in M \times W} x_{ij}\\
\text{subject to constraints \ref{constraintA}-\ref{constraintD}}
\end{align*}

But once again, the constraint (\ref{constraintD}) is not in standard form. We can go through a similar exercise as before, creating a large single matrix that allows us to combine all constraints into a single constraint. 

For instance, consider the first sum in constraint (\ref{constraintD}):
\[
\sum_{j >_m w}x_{m,j}
\]

For each man $m_i$, define a relative ranking indicator matrix $R^i \in \mathbb{R}^{N \times N}$ such that $R^i_{k,l}=1 \Rightarrow w_k \geq_m w_l$, else $R^i_{k,l}=0$. That is, $R^i_{k,l}$ indicates whether man $m_i$ prefers woman $w_k$ over $w_l$. Then form a block matrix $R=\text{diag}(R^1, R^2, ..., R^N) \in \mathbb{R}^{N^2 \times N^2}$, which allows us to write:
\[
\sum_{j >_m w}x_{m,j} \geq z, \> \forall(m,w) \rightarrow R\textbf{x} \geq \textbf{z}
\]

For brevity, and because the remainder of the reduction to standard form is rather tedious, we'll cut right to the chase at this point.

Note that whereas constraints (\ref{constraintA}) and (\ref{constraintB}) each introduced $N$ decision variables into the dual assignment problem (which we called $u_i$ and $v_j$), constraint (\ref{constraintD}) will introduce an \textit{additional} $N^2$ decision variables! Further, since the inequality on constraint (\ref{constraintD}) is $\geq$, we can flip this and carry on by multiplying the inequality through by $-1$, which will have the effect of bringing in the new decision variables as subtractions in the dual marriage problem.

This results in the following dual problem, with a third decision variable $\gamma$ introduced to represent the constraints from (\ref{constraintD}):

\begin{align*}
    \text{Minimize } &\sum_{i\in M}\alpha_i + \sum_{j\in W}\beta_j - \sum_{(i,j) \in M \times W} \gamma_{i,j}\\
    \text{s.t. }&\alpha_m + \beta_w - \sum_{j <_m w} \gamma_{m,j} - \sum_{i \leq_w m} \gamma_{i,w}  \geq 1 \quad \forall (m,w) \in M \times W\\
    & u_i, v_j, \gamma_{i,j} \geq 0
\end{align*}

Note that the third term in constraint (\ref{constraintD}) has been absorbed into the second summation in the dual problem constraint by replacing $<_w$ with $\leq_w$. 
Interpretation of the dual variables in the stable marriage problem is difficult, and I was unable to find any attempt to assign meaning to them. For a discussion of the difficulty in interpreting the dual variables, see section 3.2.2 of \cite{wangthesis}.


\section{Example}
We can work through the stable marriage problem in the linear programming setting by following the example, presented in the spirit of \cite{maritalBliss}.

Consider an $N=3$ instance of the stable marriage problem. Let the men's $P_M$ and women's $P_W$ preference lists be given by:
\[
P_M = \begin{bmatrix}
3 & 1  & 2\\
1&3&2\\
3&2&1
\end{bmatrix}
\qquad P_W = \begin{bmatrix}
3&1&2\\
2&3&1\\
3&2&1
\end{bmatrix}
\]

The constraints can be written exhaustively as follows:
\begin{align*}
    x_{1,1} + x_{1,2} + x_{1,3} \leq 1 \tag{A1}\\
    x_{2,1} + x_{2,2} + x_{2,3} \leq 1 \tag{A2}\\
    x_{3,1} + x_{3,2} + x_{3,3} \leq 1 \tag{A3} \label{constraintA3}\\
    x_{1,1} + x_{2,1} + x_{3,1} \leq 1 \tag{B1}\\
    x_{1,2} + x_{2,2} + x_{3,2} \leq 1 \tag{B2}\\
    x_{1,3} + x_{2,3} + x_{3,3} \leq 1 \tag{B3} \label{constraintB3}\\
    (x_{1,3}) + (x_{3,1}) + x_{1,1} \geq 1 \tag{D[1,1]}\\
    (x_{1,1} + x_{1,3}) + (x_{2,2} + x_{3,2}) + x_{1,2} \geq 1 \tag{D[1,2]}\\
    (0) + (x_{2,3} + x_{3,3}) + x_{1,2} \geq 1 \tag{D[1,3]}\\
    (0) + (x_{1,1} + x_{3,1}) + x_{2,1} \geq 1 \tag{D[2,1]}\\
    (x_{2,1} + x_{2,3}) + (0) + x_{2,2} \geq 1 \tag{D[2,2]}\\
    (x_{2,1}) + (x_{3,3}) + x_{2,3} \geq 1 \tag{D[2,3]}\\
    (x_{3,2} + x_{3,3}) + (0) + x_{3,1} \geq 1 \tag{D[3,1]}\\
    (x_{3,3}) + (x_{2,2}) + x_{3,2} \geq 1 \tag{D[3,2]}\\
    (0) + (0) + x_{3,3} \geq 1 \tag{D[3,3]} \label{constraintD33}
\end{align*}

And subject to those, we wish to maximize the objective function:
\[
\text{Maximize:  } p = x_{1,1} + x_{1,2} + x_{1,3} + x_{2,1} + x_{2,2} + x_{2,3} + x_{3,1} + x_{3,2} + x_{3,3} 
\]

We can follow our nose through a brute-force solution to this problem by first noticing the rather convenient fact that, since man $m_3$ and woman $w_3$ both prefer each other to any others, they must be matched according to constraint (\ref{constraintD33})! That is, we know $x_{3,3} \geq 1$. Given the non-negativity constraint and constraints (\ref{constraintA3}) and (\ref{constraintB3}), we can also immediately identify that $x_{3,1}=x_{3,2}=x_{1,3}=x_{2,3}=0$, and that $x_{3,3}=1$. At this point our unused constraints now look like:
\begin{align*}
    x_{1,1} + x_{1,2} \leq 1 \tag{A'1}\\
    x_{2,1} + x_{2,2} \leq 1 \tag{A'2} \label{constraintA'2}\\
    x_{1,1} + x_{2,1} \leq 1 \tag{B'1}\\
    x_{1,2} + x_{2,2}\leq 1 \tag{B'2} \label{constraintB'2}\\
    x_{1,1} \geq 1 \tag{D'[1,1]}\\
    (x_{1,1}) + (x_{2,2}) + x_{1,2} \geq 1 \tag{D'[1,2]}\\
    (0) + (1) + x_{1,2} \geq 1 \tag{D'[1,3]}\\
    (0) + (x_{1,1}) + x_{2,1} \geq 1 \tag{D'[2,1]}\\
    (x_{2,1}) + (0) + x_{2,2} \geq 1 \tag{D'[2,2]}\\
    (x_{2,1}) + (1) \geq 1 \tag{D'[2,3]}\\
    (1) + (0) \geq 1 \tag{D'[3,1]}\\
    (0) + (x_{2,2})  \geq 1 \tag{D'[3,2]} \label{constraintD'32}
\end{align*}

From here, everything quickly collapses to find $x_{1,1}=1$, $x_{2,2}=1$, and all others equal $0$.

We are then left with an incidence vector $x = $[1 0 0 0 1 0 0 0 1], or a matching of $(1,1)$, $(2,2)$, and $(3,3)$. It can be easily verified that this matching is stable, and in fact we know that it \textit{must} be stable since it was constructed using the constraints that guarantee a stable matching.

From the other direction, we can examine the dual for this problem. The dual is not so easy to solve by inspection as the primal was for this contrived case, so here we only display the dual objective and constraints for completeness. We wish to minimize the dual objective function
\begin{align*}
\text{Minimize:  } &d = \alpha_1 + \alpha_2 + \alpha_3 + \beta_1 + \beta_2 + \beta_3\\
&\quad - \gamma_{1,1} - \gamma_{1,2} - \gamma_{1,3} - \gamma_{2,1} - \gamma_{2,2} - \gamma_{2,3} - \gamma_{3,1} - \gamma_{3,2} - \gamma_{3,3}
\end{align*}

subject to:

\begin{align*}
    \alpha_1 + \beta_1 - (\gamma_{1,2}) - (\gamma_{1,1} + \gamma_{2,1}) \geq 1\\
    \alpha_1 + \beta_2 - (0) - (\gamma_{1,2}) \geq 1\\
    \alpha_1 + \beta_3 - (\gamma_{1,1} + \gamma_{1,2}) - (\gamma_{1,3}) \geq 1\\
    \alpha_2 + \beta_1 - (\gamma_{2,2} + \gamma_{2,3}) - (\gamma_{2,1}) \geq 1\\
    \alpha_2 + \beta_2 - (0) - (\gamma_{1,2} + \gamma_{2,2} + \gamma_{3,2}) \geq 1\\
    \alpha_2 + \beta_3 - (\gamma_{2,2}) - (\gamma_{1,3} + \gamma_{2,3}) \geq 1\\
    \alpha_3 + \beta_1 - (0) - (\gamma_{1,1} + \gamma_{2,1} + \gamma_{3,1}) \geq 1\\
    \alpha_3 + \beta_2 - (\gamma_{3,1}) - (\gamma_{1,2} + \gamma_{3,2}) \geq 1\\
    \alpha_3 + \beta_3 - (\gamma_{3,1} + \gamma_{3,2}) - (\gamma_{1,3} + \gamma_{2,3} + \gamma_{3,3}) \geq 1\\
\end{align*}

\bibliographystyle{unsrt}
\bibliography{refs}
\end{document}