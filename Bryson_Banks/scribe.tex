\documentclass[twoside]{article}
\usepackage{graphics}
\setlength{\oddsidemargin}{0.25 in}
\setlength{\evensidemargin}{-0.25 in}
\setlength{\topmargin}{-0.6 in}
\setlength{\textwidth}{6.5 in}
\setlength{\textheight}{8.5 in}
\setlength{\headsep}{0.75 in}
\setlength{\parindent}{0 in}
\setlength{\parskip}{0.1 in}

%
% The following commands set up the lecnum (lecture number)
% counter and make various numbering schemes work relative
% to the lecture number.
%
\newcounter{lecnum}
\renewcommand{\thepage}{\thelecnum-\arabic{page}}
\renewcommand{\thesection}{\thelecnum.\arabic{section}}
\renewcommand{\theequation}{\thelecnum.\arabic{equation}}
\renewcommand{\thefigure}{\thelecnum.\arabic{figure}}
\renewcommand{\thetable}{\thelecnum.\arabic{table}}

%
% The following macro is used to generate the header.
%
\newcommand{\lecture}[4]{
   \pagestyle{myheadings}
   \thispagestyle{plain}
   \newpage
   \setcounter{lecnum}{#1}
   \setcounter{page}{1}
   \noindent
   \begin{center}
   \framebox{
      \vbox{\vspace{2mm}
    \hbox to 6.28in { {\bf EE 382V: Social Computing
                        \hfill Fall 2018} }
       \vspace{4mm}
       \hbox to 6.28in { {\Large \hfill Lecture #1: #2  \hfill} }
       \vspace{2mm}
       \hbox to 6.28in { {\it Lecturer: #3 \hfill Scribe: #4} }
      \vspace{2mm}}
   }
   \end{center}
   \markboth{Lecture #1: #2}{Lecture #1: #2}
   %{\bf Disclaimer}: {\it These notes have not been subjected to the
   %usual scrutiny reserved for formal publications.  They may be distributed
   %outside this class only with the permission of the Instructor.}
   \vspace*{4mm}
}

%
% Convention for citations is authors' initials followed by the year.
% For example, to cite a paper by Leighton and Maggs you would type
% \cite{LM89}, and to cite a paper by Strassen you would type \cite{S69}.
% (To avoid bibliography problems, for now we redefine the \cite command.)
% Also commands that create a suitable format for the reference list.
\renewcommand{\cite}[1]{[#1]}
\def\beginrefs{\begin{list}%
        {[\arabic{equation}]}{\usecounter{equation}
         \setlength{\leftmargin}{2.0truecm}\setlength{\labelsep}{0.4truecm}%
         \setlength{\labelwidth}{1.6truecm}}}
\def\endrefs{\end{list}}
\def\bibentry#1{\item[\hbox{[#1]}]}

%Use this command for a figure; it puts a figure in wherever you want it.
%usage: \fig{NUMBER}{SPACE-IN-INCHES}{CAPTION}
\newcommand{\fig}[3]{
			\vspace{#2}
			\begin{center}
			Figure \thelecnum.#1:~#3
			\end{center}
	}
% Use these for theorems, lemmas, proofs, etc.
\newtheorem{theorem}{Theorem}[lecnum]
\newtheorem{lemma}[theorem]{Lemma}
\newtheorem{proposition}[theorem]{Proposition}
\newtheorem{claim}[theorem]{Claim}
\newtheorem{corollary}[theorem]{Corollary}
\newtheorem{definition}[theorem]{Definition}
\newenvironment{proof}{{\bf Proof:}}{\hfill\rule{2mm}{2mm}}

% **** IF YOU WANT TO DEFINE ADDITIONAL MACROS FOR YOURSELF, PUT THEM HERE:

\begin{document}
%FILL IN THE RIGHT INFO.
%\lecture{**LECTURE-NUMBER**}{**DATE**}{**LECTURER**}{**SCRIBE**}
\lecture{29}{November 10}{Vijay Garg}{Bryson Banks}
%\footnotetext{These notes are partially based on those of Nigel Mansell.}

% **** YOUR NOTES GO HERE:

% Some general latex examples and examples making use of the
% macros follow.  
%**** IN GENERAL, BE BRIEF. LONG SCRIBE NOTES, NO MATTER HOW WELL WRITTEN,
%**** ARE NEVER READ BY ANYBODY.
\section{Answers to Problems on Auctions}
The following two exercises focus on second-price, sealed-bid auctions where a seller is selling some object that bidders each have their own independent, private valuations for.

\subsection{Exercise 9-3}
{\bf(a)} Four possible outcomes of two bidders having valuations 0 or 1:

\begin{itemize}
  \item $(v_{1}, v_{2}) = (0, 0)$: Bidder 1 or 2 wins and pays 0.
  \item $(v_{1}, v_{2}) = (0, 1)$: Bidder 2 wins and pays 0.
  \item $(v_{1}, v_{2}) = (1, 0)$: Bidder 1 wins and pays 0.
  \item $(v_{1}, v_{2}) = (1, 1)$: Bidder 1 or 2 wins and pays 1.
\end{itemize}

If we equally weight the probability of each of the four possible outcomes above, then the probability of each outcome is $\frac{1}{4}$. With this knowledge, we can then show the seller's expected revenue ($ER_{S}$) is also $\frac{1}{4}$.

$$ER_{S} = \frac{1}{4}(0) + \frac{1}{4}(0) + \frac{1}{4}(0) + \frac{1}{4}(1) = \frac{1}{4}$$

\vspace{10mm}
{\bf(b)} Eight possible outcomes of three bidders having valuations 0 or 1:

\begin{itemize}
  \item $(v_{1}, v_{2}, v_{3}) = (0, 0, 0)$: Bidder 1, 2, or 3 wins and pays 0.
  \item $(v_{1}, v_{2}, v_{3}) = (0, 0, 1)$: Bidder 3 wins and pays 0.
  \item $(v_{1}, v_{2}, v_{3}) = (0, 1, 0)$: Bidder 2 wins and pays 0.
  \item $(v_{1}, v_{2}, v_{3}) = (0, 1, 1)$: Bidder 2 or 3 wins and pays 1.
  \item $(v_{1}, v_{2}, v_{3}) = (1, 0, 0)$: Bidder 1 wins and pays 0.
  \item $(v_{1}, v_{2}, v_{3}) = (1, 0, 1)$: Bidder 1 or 3 wins and pays 1.
  \item $(v_{1}, v_{2}, v_{3}) = (1, 1, 0)$: Bidder 1 or 2 wins and pays 1.
  \item $(v_{1}, v_{2}, v_{3}) = (1, 1, 1)$: Bidder 1, 2, or 3 wins and pays 1.
\end{itemize}

If we equally weight the probability of each of the eight possible outcomes above, then the probability of each outcome is $\frac{1}{8}$. With this knowledge, we can then show the seller's expected revenue ($ER_{S}$) has increased to $\frac{1}{2}$ with the additional bidder.

$$ER_{S} = \frac{1}{8}(0) + \frac{1}{8}(0) + \frac{1}{8}(0) + \frac{1}{8}(1) + \frac{1}{8}(0) + \frac{1}{8}(1) + \frac{1}{8}(1) + \frac{1}{8}(1) = \frac{4}{8} = \frac{1}{2}$$

\vspace{10mm}
{\bf(c)} Comparing the solutions for (a) and (b), we see that going from two bidders to three bidders increased the seller's expected revenue from $\frac{1}{4}$ to $\frac{1}{2}$. If we were to add in an additional fourth bidder, we would again expect to see the seller's expected revenue increase. The same again with a fifth bidder, and a sixth, and so on.

To prove this, we note that the seller will earn 0 anytime less than two bidders bid 1, and 1 otherwise. Thus, in this particular case the seller's expected revenue happens to be equal to the probability that at least two bidders bid 1, or 1 minus the probability that 1 or 0 bidders bid 1.

$$ER_{S} = P(2\>or\>more) = 1 - P(1\>or\>0)$$

$P(1\>or\>0)$ is equal to the number of outcomes where 1 or 0 bidders bid 1 divided by the total number of possible outcomes. Let $n$ be the number of bidders. Then, there are n different outcomes where a single bidder bids 1, and 1 outcome where every bidder bids 0. Additionally, since each bidder has only 2 possible choices, we see there are $2^n$ total possible outcomes.

$$P(1\>or\>0) = \frac{number\>of\>outcomes\>where\>1\>or\>0\>bidders\>bid\>1}{total\>number\>of\>possible\>outcomes} = \frac{n + 1}{2^n}$$.

After substituting, we get a function of $n$, the total number of bidders, to easily calculate the seller's expected revenue for any positive integer number of bidders.

$$ER_{S}(n) = 1 - \frac{n + 1}{2^n}$$

Analyzing the function, we see as n grows, so too does $ER_{S}$. We also see as n grows to infinity, $ER_{S}$ converges to 1, never actually reaching 1.

$$ER_{S}(1) = 0, \;\;ER_{S}(2) = \frac{1}{4}, \;\;ER_{S}(3) = \frac{1}{2}, \;\;ER_{S}(4) = \frac{11}{16}, \;\;ER_{S}(5) = \frac{13}{16}, \;\;\ldots$$

\newpage
\subsection{Exercise 9-4}
{\bf(a)} We know a always bids its valuation, but b sometimes makes a mistake with its bid. When b's actual valuation is 0, half the time it bids 1 mistakenly. Below we outline the possible outcomes of valuations and bids.

\begin{itemize}
  \item $(v_{a}, v_{b}) = (0, 0)$:
  \begin{itemize}
    \item {\bf Outcome 1.} $(b_{a}, b_{b}) = (0, 0)$: Bidder a ($50\%$) or b ($50\%$) wins and pays 0.
    \item {\bf Outcome 2.} $(b_{a}, b_{b}) = (0, 1)$: {\bf Mistake bid.} Bidder b wins and pays 0.
  \end{itemize}
  \item $(v_{a}, v_{b}) = (0, 1)$:
  \begin{itemize}
    \item {\bf Outcome 3.} $(b_{a}, b_{b}) = (0, 1)$: Bidder b wins and pays 0.
  \end{itemize}
  \item $(v_{a}, v_{b}) = (1, 0)$: Bidder 1 wins and pays 0.
  \begin{itemize}
    \item {\bf Outcome 4.} $(b_{a}, b_{b}) = (1, 0)$: Bidder a wins and pays 0.
    \item {\bf Outcome 5.} $(b_{a}, b_{b}) = (1, 1)$: {\bf Mistake bid.} Bidder a ($50\%$) or b ($50\%$) wins and pays 1.
  \end{itemize}
  \item $(v_{a}, v_{b}) = (1, 1)$:
  \begin{itemize}
    \item {\bf Outcome 6.} $(b_{a}, b_{b}) = (1, 1)$: Bidder a or b wins and pays 1.
  \end{itemize}
\end{itemize}

Given a bids 0 (Outcomes 1, 2, and 3), we see there is a $\frac{1}{8}$ probability it wins and pays 0, and a 0 probability it wins and pays 1. To calculate these values note that given a bids 0, outcomes 1 and 2 each have a $\frac{1}{4}$ probability of occurrence, while outcome 3 has a $\frac{1}{2}$ probability of occurrence.

$$P(a\>wins\>and\>pays\>0\>|\>a\>bids\>0) = \frac{1}{4}(\frac{1}{2}) + \frac{1}{4}(0) + \frac{1}{2}(0) = \frac{1}{8}$$

$$P(a\>wins\>and\>pays\>1\>|\>a\>bids\>0) = \frac{1}{4}(0) + \frac{1}{4}(0) + \frac{1}{2}(0) = 0$$

Given a bids 1 (Outcomes 4, 5, and 6), we see there is a $\frac{1}{4}$ probability it wins and pays 0, and a $\frac{3}{8}$ probability it wins and pays 1. To calculate these values note that given a bids 1, outcomes 4 and 5 each have a $\frac{1}{4}$ probability of occurrence, while outcome 6 has a $\frac{1}{2}$ probability of occurrence.

$$P(a\>wins\>and\>pays\>0\>|\>a\>bids\>1) = \frac{1}{4}(1) + \frac{1}{4}(0) + \frac{1}{2}(0) = \frac{1}{4}$$

$$P(a\>wins\>and\>pays\>1\>|\>a\>bids\>1) = \frac{1}{4}(0) + \frac{1}{4}(\frac{1}{2}) + \frac{1}{2}(\frac{1}{2}) = \frac{3}{8}$$

To determine if remaining true is still the dominant strategy for a given b's mistake bids, for each of the possible true valuations for a, we will  use the probabilities calculated above to analyze whether staying true or lying is better. 

\newpage
{\bf Case that a's true valuation is 0:}

When a's true valuation is 0, by staying true and bidding 0 we see using the probabilities above it only has a $\frac{1}{8}$ probability of winning and paying its valuation or less, but has 0 probability of winning and having to pay more than its valuation.

If instead it lies and bids 1, we see it has an increased $\frac{1}{4}$ probability of winning and paying its valuation or less, but it also now has a $\frac{3}{8}$ probability of winning and having to pay more than its valuation, which makes this strategy inadvisable.

Thus, it is better to stay true given a's valuation is 0.

{\bf Case that a's true valuation is 1:}

When a's true valuation is 1, by staying true and bidding 1 we see using the probabilities above it has a $\frac{1}{4} + \frac{3}{8} = \frac{5}{8}$ probability of winning and paying its valuation or less.

If instead it lies and bids 0, we see it only has a $\frac{1}{8}$ probability of winning and paying its valuation or less.

Thus, it is better to stay true given a's valuation is 1.

{\bf Conclusion:}

In each possible valuation for a, staying true remains the dominant strategy for a even considering it is aware of the mistake bids made by b.

\vspace{10mm}
{\bf(b)} If we equally weight the probability of each of the four possible valuations above at $\frac{1}{4}$, and equally the possible bids in each valuation case, we can then show the seller's expected revenue ($ER_{S}$) is $\frac{3}{8}$.

$$ER_{S} = \frac{1}{4}(\frac{1}{2}(0) + \frac{1}{2}(0)) + \frac{1}{4}(0) + \frac{1}{4}(\frac{1}{2}(0) + \frac{1}{2}(1)) + \frac{1}{4}(1) = \frac{1}{8} + \frac{1}{4} =  \frac{3}{8}$$



\end{document}





