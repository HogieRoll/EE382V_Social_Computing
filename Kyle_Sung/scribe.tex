%
% This is the LaTeX template file for lecture notes for EE 382C/EE 361C.
%
% To familiarize yourself with this template, the body contains
% some examples of its use.  Look them over.  Then you can
% run LaTeX on this file.  After you have LaTeXed this file then
% you can look over the result either by printing it out with
% dvips or using xdvi.
%
% This template is based on the template for Prof. Sinclair's CS 270.

\documentclass[twoside]{article}
\usepackage{graphics}
\usepackage{tikz}
\usetikzlibrary{positioning,chains,fit,shapes,calc}
\setlength{\oddsidemargin}{0.25 in}
\setlength{\evensidemargin}{-0.25 in}
\setlength{\topmargin}{-0.6 in}
\setlength{\textwidth}{6.5 in}
\setlength{\textheight}{8.5 in}
\setlength{\headsep}{0.75 in}
\setlength{\parindent}{0 in}
\setlength{\parskip}{0.1 in}

\definecolor{myblue}{RGB}{80,80,160}
\definecolor{mygreen}{RGB}{80,160,80}

%
% The following commands set up the lecnum (lecture number)
% counter and make various numbering schemes work relative
% to the lecture number.
%
\newcounter{lecnum}
\newcounter{qnum}
\renewcommand{\thepage}{\thelecnum-\arabic{page}}
\renewcommand{\thesection}{\thelecnum.\arabic{qnum}}
\renewcommand{\theequation}{\thelecnum.\arabic{equation}}
\renewcommand{\thefigure}{\thelecnum.\arabic{figure}}
\renewcommand{\thetable}{\thelecnum.\arabic{table}}

%
% The following macro is used to generate the header.
%
\newcommand{\lecture}[4]{
   \pagestyle{myheadings}
   \thispagestyle{plain}
   \newpage
   \setcounter{lecnum}{#1}
   \setcounter{page}{1}
   \setcounter{qnum}{8}
   \noindent
   \begin{center}
   \framebox{
      \vbox{\vspace{2mm}
    \hbox to 6.28in { {\bf EE 382V: Social Computing
                        \hfill Fall 2018} }
       \vspace{4mm}
       \hbox to 6.28in { {\Large \hfill #2  \hfill} }
       \vspace{2mm}
       \hbox to 6.28in { { \hfill Scribe: #4} }
      \vspace{2mm}}
   }
   \end{center}
   \markboth{#2}{#2}
   %{\bf Disclaimer}: {\it These notes have not been subjected to the
   %usual scrutiny reserved for formal publications.  They may be distributed
   %outside this class only with the permission of the Instructor.}
   \vspace*{4mm}
}

%
% Convention for citations is authors' initials followed by the year.
% For example, to cite a paper by Leighton and Maggs you would type
% \cite{LM89}, and to cite a paper by Strassen you would type \cite{S69}.
% (To avoid bibliography problems, for now we redefine the \cite command.)
% Also commands that create a suitable format for the reference list.
\renewcommand{\cite}[1]{[#1]}
\def\beginrefs{\begin{list}%
        {[\arabic{equation}]}{\usecounter{equation}
         \setlength{\leftmargin}{2.0truecm}\setlength{\labelsep}{0.4truecm}%
         \setlength{\labelwidth}{1.6truecm}}}
\def\endrefs{\end{list}}
\def\bibentry#1{\item[\hbox{[#1]}]}



%Use this command for a figure; it puts a figure in wherever you want it.
%usage: \fig{NUMBER}{SPACE-IN-INCHES}{CAPTION}
\newcommand{\fig}[3]{
			\vspace{#2}
			\begin{center}
			Figure \thelecnum.#1:~#3
			\end{center}
	}
% Use these for theorems, lemmas, proofs, etc.
\newtheorem{theorem}{Theorem}[lecnum]
\newtheorem{lemma}[theorem]{Lemma}
\newtheorem{proposition}[theorem]{Proposition}
\newtheorem{claim}[theorem]{Claim}
\newtheorem{corollary}[theorem]{Corollary}
\newtheorem{definition}[theorem]{Definition}
\newenvironment{proof}{{\bf Proof:}}{\hfill\rule{2mm}{2mm}}

% **** IF YOU WANT TO DEFINE ADDITIONAL MACROS FOR YOURSELF, PUT THEM HERE:

\begin{document}
%FILL IN THE RIGHT INFO.
%\lecture{**LECTURE-NUMBER**}{**DATE**}{**LECTURER**}{**SCRIBE**}
\lecture{6}{Chapter 6: Excercise 8, 9}{Vijay Garg}{Kyle Sung}
%\footnotetext{These notes are partially based on those of Nigel Mansell.}

% **** YOUR NOTES GO HERE:

% Some general latex examples and examples making use of the
% macros follow.  
%**** IN GENERAL, BE BRIEF. LONG SCRIBE NOTES, NO MATTER HOW WELL WRITTEN,
%**** ARE NEVER READ BY ANYBODY.
\section{Problem Overview}
The problem is a coordination game with the payoff matrix 

\begin{tabular}{ |c|c| } 
 \hline
 1,1 & 0,0 \\ 
\hline
 0,0 & 4,4 \\ 
 \hline
\end{tabular}

\subsection{Part a}
Q: Find all pure-strategy Nash equilibria for this game. 

A: (U, L) and (D, R) are both NE.

\subsection{Part b}
Q: Find the mix-strategy NE, and give an explanation.

A:

A plays U with prob p

B plays L with prob q

Expected payoff for A to play U: 
q*1 + (1-q)*0 = q

Expected payoff for A to play D: 
q*0 + (1-q)*4 = 4 - 4q 

Indifference for B:

q = 4 - 4q, 5q = 4. q = 4/5. 

Expected payoff for B to play L: 

p*1 + (1-p)*0 = p

Expected payoff for B to play R: 

p*0 + (1-p)*4 = 4 - 4p 

Indifference for A:

p = 4 - 4p, 5p = 4, p = 4/5.

Mixed strategy NE: ((4/5,1/5), (4/5,1/5))
When player A expect player B to play L with the probability of 4/5, then A will try to match the probability as well.


\subsection{Part c}
Q: With Schelling's focal point idea, what equilibrium do you think is the best prediction of how the game will be played?

A: When a Schelling’s focal point can be expected, player A would expect player B to play R, since it has the better payoff of the two NE. Likewise for player B to expect A to play D. Thus (D, R) would be the converged focal point.

\section{Problem Overview}
Find all Nash equilibria given the payoff matrices:

\subsection{Part a}
\begin{tabular}{ |c|c| } 
 \hline
 8,4 & 5,5 \\ 
\hline
 3,3 & 4,8 \\ 
 \hline
\end{tabular}

Pure strategy NE:

(U,R)

Mixed strategy NE:

A plays U with prob p

B plays L with prob q

Expected payoff for A to play U:

q*8 + (1-q)*5 = 5 + 3q

Expected payoff for A to play D:

q*3 + (1-q)*4 = 4 - q

Indifference for B:

5 + 3q = 4 - q, 4q = -1. q = -1⁄4. i.e. B always plays R 

Expected payoff for B to play L:

p*4 + (1-p)*3 = 1 + p

Expected payoff for B to play R:

p*5 + (1-p)*8 = 8 - 3p

Indifference for A:

1 + p = 8 - 3p, 4p = 7, p = 7/4. i.e. A always plays U

This just shows that if a pure strategy NE exists, the mixed strategy NE would be the same (i.e. U,R in this case)

\subsection{Part b}
\begin{tabular}{ |c|c| } 
 \hline
 0,0 & -1,1 \\ 
\hline
 -1,1 & 2,-2 \\ 
 \hline
\end{tabular}

Pure strategy NE:

Does not exist

Mixed strategy NE:

A plays U with prob p

B plays L with prob q

Expected payoff for A to play U:

q*0 + (1-q)*-1 = -1 + q

Expected payoff for A to play D:

q*-1 + (1-q)*2 = -q + 2 - 2q = 2 - 3q

Indifference for B:

-1+q=2-3q,4q =3,q=3/4.

Expected payoff for B to play L:

p*0 + (1-p)*1 = 1 - p

Expected payoff for B to play R:

p*1 + (1-p)*-2 = -2 + 3p

Indifference for A:

1 - p = -2 + 3p, 3 = 4p, p = 3⁄4.

((3/4, 1/4), (3/4, 1/4))

\end{document}